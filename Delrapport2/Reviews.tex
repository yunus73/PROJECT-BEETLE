\documentclass[a4paper]{article}
\usepackage[utf8]{inputenc}
\usepackage[danish]{babel}

\usepackage{amsmath}
\usepackage{amsfonts}
\usepackage{amssymb}
\usepackage{graphicx}


% Ved at bruge kommandoen \newcommand kan man forkorte kommandoer eller ændre dem til noget mere passende.
\newcommand{\setR}{\mathbb{R}}
\newcommand{\setZ}{\mathbb{Z}}
\newcommand{\setN}{\mathbb{N}}
\newcommand{\setF}{\mathbb{F}}
\newcommand{\lra}{\Leftrightarrow}
\newcommand{\ra}{\Rightarrow}
\newcommand{\ac}{\textasciicircum}
\newcommand{\uuline}[1]{\underline{\underline{#1}}}
\newcommand{\bpm}{\begin{pmatrix}}
\newcommand{\epm}{\end{pmatrix}}


\title{Reviews}
\author{Project Beetle}
\begin{document}
\maketitle
\section{Designing for usability - J.D. Gould and C. Lewis}
The text is based on the premise of  three simple principles: Early focus on users, emphirical measurement and iterative design. The text is about system designs and how in the modern day everyone may find change very drastic. The designers and developers job is to make everything easy accesible, and to make the system as something which can turn into a successor to old habits. \\
The fact that the article is from 1985 would in some people's minds be a distress signal as the paper is 30 years old. What could designers 30 years ago possibly now about designing systems, considering the technology we work with today? Well to our surprise the paper was very valid even today. The principles which the paper states are things people may feel are intuitive and obvious. Even we felt it was quite obvious. But either way it is not to be forgotten that when you begin designing, you need to know the users who are going to use it. The developer needs to think as a user, and not as a developer. Making something that seems easy for you might not seem as easy for others. The paper also emphasizes the use of emphirical measurements. Developers tend to forget the purpose of testing it in public. Every type of person should be involved in the testing again because of the fact that not all people may find using the given system easy. Presenting the system for new users/testers should also be done in a manner in which the normal citizen can relate to why the system provides improvements. It is naive to only look at it in a rational matter, because doing so doesn't involve users at all. It is the users after all who are going to use it. One cannot say that the users don't know what they want, they just need help to understand it. Why should the school system in Denmark start using programs like TI-Nspire and Maple, when a calculator and paper has worked for centuries? The argumentation and advantages needs to be pointed out. Another good point the paper mentions is promises. Promising voice recognition and touch screen doesn't exactly mean that the device/system will be of good quality. The design must be iterative, their is no advantage in locking the design at one state, this is only stubborn. Everything can be improved.
The paper in short terms focuses on the involvement of users, and how the developers can ínvolve more people. Changing habits is hard, but that doesn't mean it is impossible. 
\\
\\
This paper can be related to our own project involving the University of Hamburg and their wish for a search engine. Our approach so far hasn't really followed the principles to the fullest, but since we are so early in the developing process it doesn't matter that much as of yet. What we have done, is kept a constant contact with our project owner. After every session we ask the owner how he feels about our work. Our intended way of working is using the AGIL project management, where we make sure to be open to ideas, and try to implement the ideas the owner asks for. What we maybe should consider is asking other users, since the owner might feel like it's a good idea, but his customers/users might not feel the same way. Even though we're only making a search engine, which should be straight forward, we can't work with the premise that people can't have issues at this early stage. We certainly need to remember testing it with users of all sorts, going forward. User involvement is something we need to try and implement in the future work. 
\newpage

\section{A Rational Design Process - David Lorge Parnas and Paul C. Clements}
This text is about a rational approach to designing your system. The idea is that you can't achieve one perfect result, but that you can fake it. Faking isn't seen as a bad thing in this situation. The idea is to do alot of background work before the project coding is commenced. Today alot of coders usually use have a "stream of consciousness" in where they realize a idea. The idea is then executed in a "stream of execution". This is in general a bad approach because the amount of time one would use to reach a result would take to long. That is why the paper recommends first making a requirements page, in where you state what you as a developer can do, what the software has to include according to the customer. The computer specification - which OS should the e.g. software work one? Timing and accuracy constraints also help the developer set goals, and how accurate the software the person creates must be. A good idea is also in the requirements document is likely changes - this would not only make it easier for the developer long term, but also for newcomers who wish to improve the software. Another thing this paper urges new developers to do is to make sure the documentation is on point. The paper urges developers to not have poor organization - avoiding "stream of consciousness". Writing alot of text, when a shorter an more efficient sentence could be used. Using confusing terminology is also considered to be avoided - it is important to remember that even though you have lived inside this software for months, doesn't mean that newcomers will understand what you are talking about only because you do. The small details are not that important as one would think, describing the bigger picture is considered more important.
The general idea is to use clear and easy to understand terms, using figures and expressions to make it all easier for newcomers to work on. The ending documentation isn't meant to be relaxing to read, but interesting, since it should reward the reader with precise and detailed information.
\\
\\
The key element in our project is the documentation. There is a reason that we need to use more time on the document then on the code itself. The backbone to the whole project is the documentation. We try to implement as many figures as we possibly can. One thing we could work on is not using long phrases and sentences which seems to be an returning element in our documents. Something we should consider implementing in our work in the future is to make more precise requirements, as we now usually decide whilst working. If we had something to work up to it would make the process much easier, and easier to see how much we actually are missing to reach the requirements. Besides if the project is to be given to Hamburg, they would surely love to now the requirements for future developers to work and adjust according to. We should try doing more rational design, as it would prove to make our work easier in the future.


\end{document}
