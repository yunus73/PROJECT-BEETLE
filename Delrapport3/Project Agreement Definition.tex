\documentclass[12pt,a4paper]{article}

\setlength{\textwidth}{165mm}
\setlength{\textheight}{240mm}
\setlength{\parindent}{0mm} % S{\aa} meget rykkes ind efter afsnit
\setlength{\parskip}{\parsep}
\setlength{\headheight}{0mm}
\setlength{\headsep}{0mm}
\setlength{\hoffset}{-2.5mm}
\setlength{\voffset}{0mm}
\setlength{\footskip}{15mm}
\setlength{\oddsidemargin}{0mm}
\setlength{\topmargin}{0mm}
\setlength{\evensidemargin}{0mm}

\usepackage[all]{xy}
\usepackage{graphicx}    % For grafik (billederfiler)
\usepackage[T1]{fontenc} % For at blande \textsc{} med \textbf{}
\usepackage[utf8]{inputenc}
\usepackage{amsfonts,amsmath,amssymb}
\usepackage{eucal}
\usepackage[danish]{babel}
\usepackage{enumerate}  
\usepackage{hyperref}
\usepackage{url}
\usepackage{mathptmx}
\usepackage{multirow}
\usepackage[dvipsnames,usenames]{color}
\usepackage{tabularx,colortbl,xcolor}
\usepackage{listings}
\usepackage{color}
\usepackage{amsmath}
\usepackage{xcolor}

\definecolor{KU-red}{RGB}{144,26,30} 
\definecolor{dkgreen}{rgb}{0,0.6,0}
\definecolor{gray}{rgb}{0.5,0.5,0.5}
\definecolor{mauve}{rgb}{0.58,0,0.82}

\lstset{frame=tb,
  language=Java,
  aboveskip=3mm,
  belowskip=3mm,
  showstringspaces=false,
  columns=flexible,
  basicstyle={\small\ttfamily},
  numbers=none,
  numberstyle=\tiny\color{gray},
  keywordstyle=\color{blue},
  commentstyle=\color{dkgreen},
  stringstyle=\color{mauve},
  breaklines=true,
  breakatwhitespace=true,
  tabsize=3}

\DeclareSymbolFont{usualmathcal}{OMS}{cmsy}{m}{n}
\DeclareSymbolFontAlphabet{\mathcal}{usualmathcal}
\DeclareSymbolFont{letters}{OML}{txmi}{m}{it}

\DeclareMathSymbol{\alpha}{\mathord}{letters}{"0B}
\DeclareMathSymbol{\beta}{\mathord}{letters}{"0C}
\DeclareMathSymbol{\gamma}{\mathord}{letters}{"0D}
\DeclareMathSymbol{\delta}{\mathord}{letters}{"0E}
\DeclareMathSymbol{\epsilon}{\mathord}{letters}{"0F}
\DeclareMathSymbol{\zeta}{\mathord}{letters}{"10}
\DeclareMathSymbol{\eta}{\mathord}{letters}{"11}
\DeclareMathSymbol{\theta}{\mathord}{letters}{"12}
\DeclareMathSymbol{\iota}{\mathord}{letters}{"13}
\DeclareMathSymbol{\kappa}{\mathord}{letters}{"14}
\DeclareMathSymbol{\lambda}{\mathord}{letters}{"15}
\DeclareMathSymbol{\mu}{\mathord}{letters}{"16}
\DeclareMathSymbol{\nu}{\mathord}{letters}{"17}
\DeclareMathSymbol{\xi}{\mathord}{letters}{"18}
\DeclareMathSymbol{\pi}{\mathord}{letters}{"19}
\DeclareMathSymbol{\rho}{\mathord}{letters}{"1A}
\DeclareMathSymbol{\sigma}{\mathord}{letters}{"1B}
\DeclareMathSymbol{\tau}{\mathord}{letters}{"1C}
\DeclareMathSymbol{\upsilon}{\mathord}{letters}{"1D}
\DeclareMathSymbol{\phi}{\mathord}{letters}{"1E}
\DeclareMathSymbol{\chi}{\mathord}{letters}{"1F}
\DeclareMathSymbol{\psi}{\mathord}{letters}{"20}
\DeclareMathSymbol{\omega}{\mathord}{letters}{"21}
\DeclareMathSymbol{\varepsilon}{\mathord}{letters}{"22}
\DeclareMathSymbol{\vartheta}{\mathord}{letters}{"23}
\DeclareMathSymbol{\varpi}{\mathord}{letters}{"24}
\DeclareMathSymbol{\varrho}{\mathord}{letters}{"25}
\DeclareMathSymbol{\varsigma}{\mathord}{letters}{"26}
\DeclareMathSymbol{\varphi}{\mathord}{letters}{"27}
\DeclareMathSymbol{\Gamma}{\mathord}{letters}{"00}
\DeclareMathSymbol{\Delta}{\mathord}{letters}{"01}
\DeclareMathSymbol{\Theta}{\mathord}{letters}{"02}
\DeclareMathSymbol{\Lambda}{\mathord}{letters}{"03}
\DeclareMathSymbol{\Xi}{\mathord}{letters}{"04}
\DeclareMathSymbol{\Pi}{\mathord}{letters}{"05}
\DeclareMathSymbol{\Sigma}{\mathord}{letters}{"06}
\DeclareMathSymbol{\Upsilon}{\mathord}{letters}{"07}
\DeclareMathSymbol{\Phi}{\mathord}{letters}{"08}
\DeclareMathSymbol{\Psi}{\mathord}{letters}{"09}
\DeclareMathSymbol{\Omega}{\mathord}{letters}{"0A}
\DeclareMathSymbol{\upGamma}{\mathalpha}{operators}{"00}
\DeclareMathSymbol{\upDelta}{\mathalpha}{operators}{"01}
\DeclareMathSymbol{\upTheta}{\mathalpha}{operators}{"02}
\DeclareMathSymbol{\upLambda}{\mathalpha}{operators}{"03}
\DeclareMathSymbol{\upXi}{\mathalpha}{operators}{"04}
\DeclareMathSymbol{\upPi}{\mathalpha}{operators}{"05}
\DeclareMathSymbol{\upSigma}{\mathalpha}{operators}{"06}
\DeclareMathSymbol{\upUpsilon}{\mathalpha}{operators}{"07}
\DeclareMathSymbol{\upPhi}{\mathalpha}{operators}{"08}
\DeclareMathSymbol{\upPsi}{\mathalpha}{operators}{"09}
\DeclareMathSymbol{\upOmega}{\mathalpha}{operators}{"0A}

\newcommand{\hhemail}[1]{\textsf{#1}}
\newcommand{\hhurl}[1]{{\color{blue}\url{#1}}}

\begin{document}

{\huge {Project Agreement Definition}} \\
\textit{Project: Beetle for University of Hamburg developed by Yunus Emre Okutan, Enes Golic, Tor-Salve Dalsgaard and Casper Lützhøft Christensen.}\\

\textbf{Product specification}\\
The product for the University of Hamburg Entomology department will be a database containing entries of insects, a front-end search engine to allow users to search the database and retrieve information and an application to allow the desired employees to edit the database and add, update and remove entries.\\\\
\textbf{Requirements}\\
In order to develop the desired product, we require access to the University of Hamburg’s server so that we can create the database and the search engine on their website. \\
We also require the URLs to all the images of the entries they want added to the database initially, as well as the information corresponding to those.\\\\
\textbf{Payment}\\
The product is being developed as part of the course PKSU from the University of Copenhagen, as such the work committed and product developed are entirely voluntary and no payment is required from the side of the University of Hamburg.\\\\
\textbf{Terms}\\
The University of Hamburg cannot expect a fully functioning product to be delivered; they can however expect our best effort in the endeavor.\\
During development before the final product is delivered, we are in full control of the program and the University of Hamburg must in no way change, delete or in any way modify any parts of the program.\\
Once the product is delivered, it belongs entirely to the University of Hamburg. They are free to use the product for as long as they desire and can freely, in any way modify or delete it if they see fit.\\
We reserve the rights to any methods/code used in the development of the product.\\
All information, knowledge and material that is the property of the University of Hamburg that we encounter during our product-development is entirely confidential and nothing we see or read will ever be exposed to people outside the product agreement.
Given access to the database, we are in no way to edit anything not directly related to the development of our product.\footnote{Bernd Bruegge, Allen H.Dutoit-Third Edition. Object-Oriented Software Engineering. Page(568)}\\\\


\_\_\_\_\_\_\_\_\_\_\_\_\_\_\_\_\_\_\_\_ ~~~~~~~~~~~~~\_\_\_\_\_\_\_\_\_\_\_\_\_\_\_\_\_\_\_\_\_\_\_\_\_\_\_\_\_\_\_\_\_\_\_\_\_\_\_\_\_\_\_\_\\
Client~~~~~~~~~~~~~~~~~~~~~~~~~~~~~~~~~~~~~~~~~~~~~Developers       
                                                   
\end{document}