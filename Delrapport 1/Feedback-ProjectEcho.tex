\documentclass[a4paper]{article}
\usepackage[utf8]{inputenc}
\usepackage[danish]{babel}

\usepackage{amsmath}
\usepackage{amsfonts}
\usepackage{amssymb}
\usepackage{graphicx}


% Ved at bruge kommandoen \newcommand kan man forkorte kommandoer eller ændre dem til noget mere passende.
\newcommand{\setR}{\mathbb{R}}
\newcommand{\setZ}{\mathbb{Z}}
\newcommand{\setN}{\mathbb{N}}
\newcommand{\setF}{\mathbb{F}}
\newcommand{\lra}{\Leftrightarrow}
\newcommand{\ra}{\Rightarrow}
\newcommand{\ac}{\textasciicircum}
\newcommand{\uuline}[1]{\underline{\underline{#1}}}
\newcommand{\bpm}{\begin{pmatrix}}
\newcommand{\epm}{\end{pmatrix}}



\title{Project Echo - Feedback}
\author{Reviewers: Project Foxtrot}
\begin{document}
\maketitle
\section{Forord}
Vores gruppe har taget udgangspunkt i dokumentet "ReviewTeknikken" fundet på Absalon, som beskriver de grundlæggende principper bag en review. Vi vurdere derfor projektet med et neutralt synspunkt og dermed få en uvildig vurdering. 
Vi vil selvfølgelig slå ned på ting som kunne beskrives mere præcist i opgaveteksten, som vi i gruppen finder vigtige, men først og fremmest vil vi kigge på deres produkt og udfordringerne bag det.
\subsection{De positive elementer}
Umiddelbart virker projektet til at være simpelt, som i at en gruppe med lidt kode erfaring, sagtens kan lære hvordan man f.eks. kan lave en upload funktion, eller hvordan man sætter en hjemmeside op generelt. \\

Det positive ved dette projekt at læringskurven er høj. Der er mange ting gruppen kan få ud af dette projekt. Hvordan man sætter en hjemmeside op. Hvordan et stream fungere. Hvordan man opretter en upload funktion etc. Gruppen kan forvente at komme ud af dette projekt med en hel masse viden de kan bruge i fremtiden.


\subsection{De negative elementer}
I rapporten er der visse ting der er uklare eller simpelthen ikke burde være tilstede. Forsiden er tiltænkt spøgelset Kasper, gruppen består af scrum master Tor og hans slaver, men gruppen vælger stadigvæk at sige at de går efter kvalitet fremfor kvantitet.\\
Ønsket fra vores side er at se lidt mere seriøsitet. 
\\
En anden ting er at jeres opgave er skrevet op på en form for punktform, hvilket gør jeres tekst knap så flydende. Man kan argumentere for at det er kort og præcist, men visse elementer bliver bare for hurtigt vendt.\\
Hvad er formålet med hele dette projekt? Det er vel ikke bare at lave en hjemmeside? Skal det hjælpe nogle folk? 
\\
Det farlige ved dette projekt er at det godt kan blive en lang affære. Gruppen har en grundplan hvilket er helt fint, men de elementer der skal implementeres er ting der skal læses op på, studeres og prøves før man kan sætte det op på en hjemmeside for en musisk organisation. 
\\
Desuden kan det være en stor udfordring at gøre den bruger venlig set på jeres kommentarer omkring at kunden gerne selv skulle kunne bruge nogle skabeloner til at ændre/opdatere visse elementer selv. Dette kan muligvis være den sværeste del ved jeres opgave, fordi i som en gruppe sagtens kan bruge flere uger på at studere hvordan et livestream skal opkobles, og hvordan det kan interagere på en hjemmeside, hvorimod kunden kun har en kort periode at sætte sig ind i det, når produktet er udkommet og den første opdatering skal finde sted(hvor vi antager at i ikke ville være tilstede).


\end{document}
